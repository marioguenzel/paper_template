\makeatletter
\def\input@path{{resources/}}
\makeatother


% === Choose your class ===
\documentclass[conference]{IEEEtran}
% \documentclass[manuscript,acmsmall,review,anonymous,screen]{acmart}
% \documentclass[a4paper,USenglish, anonymous]{lipics-v2021}
% \documentclass[runningheads]{llncs}


% === Setting relevant information for submission ===
\newcommand{\targetconference}{RTAS 2025}
\newcommand{\pagelimit}{11}
\newcommand{\submissionnumber}{TBD}
\newcommand{\titlename}{TITLE}

\def\paperabstract{%
ABSTRACT
- introductory sentence to the topic (optional)
- describe the issue or main problem (create a need)
- one sentence detailing your approach (meet the need)
- list main research results
- finish with MAIN MESSAGE of the paper
}

% Submission flag
\newif\ifsubmission
\submissiontrue
% \submissionfalse   % uncomment to set false


% === PREAMBLES ===
% Common preamble
\usepackage[T1]{fontenc}
\usepackage[utf8]{inputenc}

\AtBeginDocument{%
	\providecommand\BibTeX{{%
			\normalfont B\kern-0.5em{\scshape i\kern-0.25em b}\kern-0.8em\TeX}}}

\usepackage{hyperref}

\let\Bbbk\relax
\usepackage{amsmath,amssymb,amsfonts} %amsmath already included
\usepackage[noend]{algpseudocode}
\usepackage{algorithmicx, algorithm}
\usepackage{graphicx}
\graphicspath{{resources/}}
\usepackage{textcomp}
\usepackage{xcolor}

\usepackage{tikz}
\usetikzlibrary{arrows,decorations.markings, positioning}
\usepackage{subcaption}
\usepackage{tabu}

% Class-specific preambles
\IfClassLoadedTF{IEEEtran}{\usepackage[english]{babel}
\usepackage{cite}

% \usepackage{lineno}
\usepackage[switch]{lineno}
\linenumbers
% fix problem with line numbering around align
\makeatletter
\let\LN@align\align
\let\LN@endalign\endalign
\renewcommand{\align}{\linenomath\LN@align}
\renewcommand{\endalign}{\LN@endalign\endlinenomath}
\makeatother}{}
\IfClassLoadedTF{acmart}{\input{resources/inputacmart1.tex}}{}
\IfClassLoadedTF{lipics-v2021}{\input{resources/inputlipics1.tex}}{}
\IfClassLoadedTF{llncs}{\usepackage{color}
\renewcommand\UrlFont{\color{blue}\rmfamily}
\urlstyle{rm}

\ifsubmission
    \usepackage{lineno}
    % \usepackage[switch]{lineno}
    \linenumbers
    % fix problem with line numbering around align
    \makeatletter
    \let\LN@align\align
    \let\LN@endalign\endalign
    \renewcommand{\align}{\linenomath\LN@align}
    \renewcommand{\endalign}{\LN@endalign\endlinenomath}
    \makeatother
\else
\fi}{}


% === COMMANDS ===
\usepackage{resources/mathcommands}

% === DRAW SCHEDULES ===
\usepackage{resources/scheduling}

% === WRITING ===
\newcommand{\papercomment}[3]{{\color{#1}#2:#3}\ClassWarning{}{There are comments left in the paper. [Remove them before submission!]}}
\newcommand{\mario}[1]{\papercomment{red}{mg}{#1}}
\newcommand{\kuan}[1]{\papercomment{blue}{kh}{#1}}
\newcommand{\jj}[1]{\papercomment{orange}{jj}{#1}}


% === PAPER SPECIFIC ===
\hyphenation{con-straints}

% ==========
\begin{document}

% === INIT AND TITLES ===
% Class-specific
\IfClassLoadedTF{IEEEtran}{\title{\titlename}
	
%% AUTHORS:
\ifsubmission
    \author{\targetconference{} Submission \#\submissionnumber{} \quad\quad\quad Pages: \pageref{last-page}/\pagelimit{}}
\else
    \author{
		\IEEEauthorblockN{First Author, Second Author and Third Author}
		\IEEEauthorblockA{TU Dortmund University\\
			Email: \{first.author, second.author, third.author\}@tu-dortmund.de}
		\and
		\IEEEauthorblockN{Other Author}
		\IEEEauthorblockA{Other University\\
			Email: other.author@other-university.de}
		}
\fi

\maketitle

\begin{abstract}
	\paperabstract
\end{abstract} }{}
\IfClassLoadedTF{acmart}{
%%
%% The "title" command has an optional parameter,
%% allowing the author to define a "short title" to be used in page headers.
\title{\titlename}

%%
%% The "author" command and its associated commands are used to define
%% the authors and their affiliations.
%% Of note is the shared affiliation of the first two authors, and the
%% "authornote" and "authornotemark" commands
%% used to denote shared contribution to the research.
\author{Ben Trovato}
\authornote{Both authors contributed equally to this research.}
\email{trovato@corporation.com}
\orcid{1234-5678-9012}
\author{G.K.M. Tobin}
\authornotemark[1]
\email{webmaster@marysville-ohio.com}
\affiliation{%
  \institution{Institute for Clarity in Documentation}
  \city{Dublin}
  \state{Ohio}
  \country{USA}
}

\author{Lars Th{\o}rv{\"a}ld}
\affiliation{%
  \institution{The Th{\o}rv{\"a}ld Group}
  \city{Hekla}
  \country{Iceland}}
\email{larst@affiliation.org}

\author{Valerie B\'eranger}
\affiliation{%
  \institution{Inria Paris-Rocquencourt}
  \city{Rocquencourt}
  \country{France}
}

\author{Aparna Patel}
\affiliation{%
 \institution{Rajiv Gandhi University}
 \city{Doimukh}
 \state{Arunachal Pradesh}
 \country{India}}

\author{Huifen Chan}
\affiliation{%
  \institution{Tsinghua University}
  \city{Haidian Qu}
  \state{Beijing Shi}
  \country{China}}

\author{Charles Palmer}
\affiliation{%
  \institution{Palmer Research Laboratories}
  \city{San Antonio}
  \state{Texas}
  \country{USA}}
\email{cpalmer@prl.com}

\author{John Smith}
\affiliation{%
  \institution{The Th{\o}rv{\"a}ld Group}
  \city{Hekla}
  \country{Iceland}}
\email{jsmith@affiliation.org}

\author{Julius P. Kumquat}
\affiliation{%
  \institution{The Kumquat Consortium}
  \city{New York}
  \country{USA}}
\email{jpkumquat@consortium.net}

%%
%% By default, the full list of authors will be used in the page
%% headers. Often, this list is too long, and will overlap
%% other information printed in the page headers. This command allows
%% the author to define a more concise list
%% of authors' names for this purpose.
\renewcommand{\shortauthors}{Trovato et al.}

%%
%% The abstract is a short summary of the work to be presented in the
%% article.
\begin{abstract}
  \paperabstract
\end{abstract}

%%
%% The code below is generated by the tool at http://dl.acm.org/ccs.cfm.
%% Please copy and paste the code instead of the example below.
%%
\begin{CCSXML}
<ccs2012>
 <concept>
  <concept_id>00000000.0000000.0000000</concept_id>
  <concept_desc>Do Not Use This Code, Generate the Correct Terms for Your Paper</concept_desc>
  <concept_significance>500</concept_significance>
 </concept>
 <concept>
  <concept_id>00000000.00000000.00000000</concept_id>
  <concept_desc>Do Not Use This Code, Generate the Correct Terms for Your Paper</concept_desc>
  <concept_significance>300</concept_significance>
 </concept>
 <concept>
  <concept_id>00000000.00000000.00000000</concept_id>
  <concept_desc>Do Not Use This Code, Generate the Correct Terms for Your Paper</concept_desc>
  <concept_significance>100</concept_significance>
 </concept>
 <concept>
  <concept_id>00000000.00000000.00000000</concept_id>
  <concept_desc>Do Not Use This Code, Generate the Correct Terms for Your Paper</concept_desc>
  <concept_significance>100</concept_significance>
 </concept>
</ccs2012>
\end{CCSXML}

\ccsdesc[500]{Do Not Use This Code~Generate the Correct Terms for Your Paper}
\ccsdesc[300]{Do Not Use This Code~Generate the Correct Terms for Your Paper}
\ccsdesc{Do Not Use This Code~Generate the Correct Terms for Your Paper}
\ccsdesc[100]{Do Not Use This Code~Generate the Correct Terms for Your Paper}

%%
%% Keywords. The author(s) should pick words that accurately describe
%% the work being presented. Separate the keywords with commas.
\keywords{Do, Not, Us, This, Code, Put, the, Correct, Terms, for,
  Your, Paper}
%% A "teaser" image appears between the author and affiliation
%% information and the body of the document, and typically spans the
%% page.
\begin{teaserfigure}
  % \includegraphics[width=\textwidth]{sampleteaser}
  \caption{Seattle Mariners at Spring Training, 2010.}
  \Description{Enjoying the baseball game from the third-base
  seats. Ichiro Suzuki preparing to bat.}
  \label{fig:teaser}
\end{teaserfigure}

\received{20 February 2007}
\received[revised]{12 March 2009}
\received[accepted]{5 June 2009}

%%
%% This command processes the author and affiliation and title
%% information and builds the first part of the formatted document.
\maketitle}{}
\IfClassLoadedTF{lipics-v2021}{\maketitle

%TODO mandatory: add short abstract of the document
\begin{abstract}
\paperabstract
\end{abstract}}{}
\IfClassLoadedTF{llncs}{\title{\titlename}
%
%\titlerunning{Abbreviated paper title}
% If the paper title is too long for the running head, you can set
% an abbreviated paper title here
%

%% AUTHORS:
\ifsubmission
    \author{\targetconference{} Submission \#\submissionnumber{} \quad\quad\quad Pages: \pageref{last-page}/\pagelimit{}}
\else
    \author{First Author\inst{1}\orcidID{0000-1111-2222-3333} \and
    Second Author\inst{2,3}\orcidID{1111-2222-3333-4444} \and
    Third Author\inst{3}\orcidID{2222--3333-4444-5555}}
    %
    \authorrunning{F. Author et al.}
    % First names are abbreviated in the running head.
    % If there are more than two authors, 'et al.' is used.
    %
    \institute{Princeton University, Princeton NJ 08544, USA \and
    Springer Heidelberg, Tiergartenstr. 17, 69121 Heidelberg, Germany
    \email{lncs@springer.com}\\
    \url{http://www.springer.com/gp/computer-science/lncs} \and
    ABC Institute, Rupert-Karls-University Heidelberg, Heidelberg, Germany\\
    \email{\{abc,lncs\}@uni-heidelberg.de}}
\fi

%
\maketitle              % typeset the header of the contribution
%
\begin{abstract}
The abstract should briefly summarize the contents of the paper in
150--250 words.

\keywords{First keyword  \and Second keyword \and Another keyword.}
\end{abstract}}{}

% === CONTENT ===

\section{Introduction}
\label{sec:introduction}

This paper has been generated using the paper template~\cite{papertemplate}.
% General introduction

% Topic of this paper

% Detailed topic / tell more details


% Contribution
\noindent\textbf{Contributions:} % short text
\begin{itemize}
    \item one
    \item two
    \item three
\end{itemize}
	
	
\section{System Model}
\label{sec:system_model}

\section{Problem Definition}
\label{sec:problem_def}

\section{Main Topic}

\begin{definition}[Test]
    Test test test
\end{definition}

\begin{figure}
    \centering
    \begin{tikzpicture}[yscale=0.5, xscale=0.4]
    \grid{0}{1}{13}{0.2}{5}
    \begin{scope}[shift={(0,2)}] % task one
    	\taskname{$\tau_1$}
    	
    	\timeline{0}{14}{}
    	% no labelling
    	
    	\releases{0,5,10}
    	\deadlines{5,10}
    	
    	\exec{0}{1}
    	\exec{5}{6}
    	\exec{10}{11}
    	
    	% no suspension
    \end{scope}
    \begin{scope}[shift={(0,0)}] % task two
    	\taskname{$\tau_2$}
    	
    	\timeline{0}{14}{}
    	\labelling{0}{13}{2}{0}
    	
    	\releases{0}
    	\deadlines{12}
    	
    	\exec{1}{2}
    	\exec{9}{10}
    	
    	\susp{2}{9}
    \end{scope}
    \end{tikzpicture}
    \caption{A nice schedule.}
    \label{fig:schedule_example}
\end{figure}



\section{Evaluation}
\label{sec:evaluation}
	

\section{Conclusion}
\label{sec:conclusion}
	

% === BIB ===
\label{last-page}
\clearpage

% Class-specific
\IfClassLoadedTF{IEEEtran}{
    \bibliographystyle{abbrv}
    \bibliography{sample}
}{}
\IfClassLoadedTF{acmart}{
        % \begin{acks}
    %     ACKS can go here
    % \end{acks}

    \bibliographystyle{ACM-Reference-Format}
    \bibliography{sample.bib}

    % \appendix
    % \section{Appendix}
}{}
\IfClassLoadedTF{lipics-v2021}{
    \bibliography{sample.bib}
    % \appendix
    % \section{Appendix}
}{}
\IfClassLoadedTF{llncs}{
    % \begin{credits}
    % \subsubsection{\ackname} 
    %     % Put acknowledgement here
    % \end{credits}
    
    \bibliographystyle{splncs04}
    \bibliography{sample}
}{}


\end{document}